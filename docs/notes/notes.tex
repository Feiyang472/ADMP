\documentclass[12pt]{extarticle}

\usepackage{physics, amsmath, amsfonts, hyperref, cleveref}

\usepackage[top = 2cm, bottom = 2cm, left = 2cm, right = 2cm]{geometry}

% \usepackage{LentSheets}

\allowdisplaybreaks

\makeatletter
% \renewcommand{\maketitle}{\bgroup\setlength{\parindent}{0pt}
%     \begin{flushleft}
%         \textbf{\Large\@title}\\
%         \@author
%     \end{flushleft}\egroup
% }
% \makeatother

\renewcommand{\bf}{\mathbf}

\title{Multipolar particle mesh Ewald programming notes}
\author{Jichen, Feiyang}

\begin{document}
    \maketitle
    \section{Background}
    \href{https://pubs.acs.org/doi/pdf/10.1021/ct5007983}{JCTC paper}

    Ewald and Particle Mesh Ewald(PME) are well-established methods for calculating long-ranged electrostatic forces.
    
    Starting with the exact electrostatic energy of a charge distribution \(\rho(\bf r)\), periodic under translations \(\bf n\)
    \[
        E = \frac{1}{2} \iint \rho(\bf r) \sum_{\bf n} \frac{\rho \pqty{ \bf r' + \bf n}}{\abs{\bf r - \bf r'}} \dd[3]{\bf r'} \dd[3]{\bf r}
    \]
    Ewald decomposes the charge density into smooth and discontinuous components
    \begin{gather*}
        \sum_{\bf n} \rho \pqty{ \bf r' + \bf n} = \underbrace{\sum_{\bf n} \tilde{\rho} \pqty{ \bf r' + \bf n}}_\text{reciprocal space part} + \underbrace{\bqty{\sum_{\bf n} \rho \pqty{ \bf r' + \bf n} - \sum_{\bf n} \tilde{\rho} \pqty{ \bf r' + \bf n}}}_ \text{real space part}\\
        E = \frac{1}{2} \iint \rho(\bf r) \sum_{\bf n} \frac{\tilde{\rho} \pqty{ \bf r' + \bf n}}{\abs{\bf r - \bf r'}} \dd[3]{\bf r'} \dd[3]{\bf r} + \frac{1}{2} \iint \rho(\bf r) \sum_{\bf n} \frac{ \rho \pqty{ \bf r' + \bf n} - \sum_{\bf n} \tilde{\rho} \pqty{ \bf r' + \bf n}}{\abs{\bf r - \bf r'}} \dd[3]{\bf r'} \dd[3]{\bf r}
    \end{gather*}
    \subsection{Spherical multipoles}
    
    \subsection{Reciprocal space Ewald}
    We approximate the Dirac delta and its STGOs with a Gaussian \(\chi\) of width \(\frac{1}{\kappa}\):
    \begin{align*}
        \delta ( \bf r - \bf R_a) &\equiv \lim_{\kappa \to  \infty} \chi (\bf r - \bf R_a)\\
        \delta_{l \mu } (\bf r - \bf R_a) &= \frac{C_{l \mu} (\grad_a)}{(2l - 1)!!} \delta (\bf r - \bf R_a)
    \end{align*}
    The charge density can be expressed as 
    \begin{align*}
        \rho (\bf r) &= \sum_{a} \sum_l^\infty q_{a, l \mu} \delta_{l \mu} \pqty{\bf r - \bf R_a}\\
        \rho (\bf r) &= \tilde{\rho} ( \bf r) + \Delta \rho(\bf r)\\
        \tilde{\rho} ( \bf r) &=  \sum_{a} \sum_{l = 0}^{l_{\max}} q_{a, l \mu} \chi_{l \mu} \pqty{\bf r - \bf R_a}
    \end{align*}
    where the correction part \(\Delta \rho (\bf r)\) is left to be treated in real space. 
    
    Periodicty is naturally incorporated into the charge density by spanning the distribution in reciprocal space \(\qty{ \bf k\; \forall\; \bf k \dotproduct \bf n = 2 \pi m, m \in \mathbb{Z}}\) 
    \begin{align*}
        \tilde{\rho} ( \bf r) &=  \sum_{a} \sum_{l = 0}^{l_{\max}} q_{a, l \mu} \frac{C_{l \mu} (\grad_a)}{(2l - 1)!!} \bra{\bf r - \bf R_a} \ket{\chi } \\
        \sum_{\bf n} \tilde{\rho} ( \bf r + \bf n) &=  \sum_{a} \sum_{l = 0}^{l_{\max}} q_{a, l \mu} \frac{C_{l \mu} (\grad_a)}{(2l - 1)!!} \bra{\bf r - \bf R_a} \frac{1}{V}\sum_{\bf k} \ket{\bf k} \bra{\bf k}  \ket{\chi } \\
        \sum_{\bf n} \tilde{\rho} ( \bf r + \bf n) &= \frac{1}{V}\sum_{\bf k}  \bra{\bf k}  \ket{\chi }  \sum_{a} \sum_{l = 0}^{l_{\max}} q_{a, l \mu} \frac{C_{l \mu} (\grad_a)}{(2l - 1)!!} \underbrace{\bra{\bf r - \bf R_a} \ket{\bf k}}_{\bra{\bf r} \ket{\bf k} \pqty{\bra{\bf R_a} \ket{\bf k} }^*  } \\
        \sum_{\bf n} \tilde{\rho} ( \bf r + \bf n) &= \frac{1}{V}\sum_{\bf k} \bra{\bf r} \ket{\bf k}  \bra{\bf k}  \ket{\chi }  \sum_{a} \sum_{l = 0}^{l_{\max}} q_{a, l \mu} \frac{C_{l \mu} (\grad_a)}{(2l - 1)!!} \bra{\bf k} \ket{\bf R_a}\\
        \sum_{\bf n} \tilde{\rho} ( \bf r + \bf n) &= \frac{1}{V}\sum_{\bf k} \bra{\bf r} \ket{\bf k}  \bra{\bf k}  \ket{\chi }  S_{\bf k} = \frac{1}{V}\sum_{\bf k} \bra{\bf k} \ket{\bf r}\bra{\chi }\ket{\bf k}  S_{\bf k}^*
    \end{align*}
    where in the last line we used realness of \(\rho\) and denoted \begin{equation}
        S_{\bf k} \equiv \sum_{a} \sum_{l = 0}^{l_{\max}} q_{a, l \mu} \frac{C_{l \mu} (\grad_a)}{(2l - 1)!!} \bra{\bf k} \ket{\bf R_a}. \label{eq:esf}
    \end{equation}
    as the structure factor. It is worth noticing \[
        \sum_{\bf n} \rho( \bf r + \bf n) = \frac{1}{V}\sum_{\bf k} \bra{\bf k} \ket{\bf r}\pqty{ \lim_{\kappa \to \infty} \bra{\chi }\ket{\bf k}}  S_{\bf k}^* = \frac{1}{V}\sum_{\bf k} \bra{\bf k} \ket{\bf r} S_{\bf k}^*
    \]
    because the Fourier transform of the Dirac delta is simply unity.

    Substituting \(\sum_{\bf n} \tilde{\rho} ( \bf r' + \bf n)\) into reciprocal space part energy, we obtain \begin{align*}
        \frac{1}{2} \iint \rho(\bf r) \sum_{\bf n} \frac{\tilde{\rho} \pqty{ \bf r' + \bf n}}{\abs{\bf r - \bf r'}} \dd[3]{\bf r'} \dd[3]{\bf r} 
        &= \frac{1}{2} \int \frac{1}{V}\sum_{\bf k} \bra{\bf k} \ket{\bf r} S_{\bf k}^* \frac{1}{V}\sum_{\bf k'}   \bra{\bf k'}  \ket{\chi }  S_{\bf k'} \underbrace{\int \frac{\bra{\bf r'} \ket{\bf k'}}{\abs{\bf r - \bf r'}} \dd[3]{\bf r'} }_{\bra{\bf r} \ket{\bf k'} \frac{4 \pi}{k^2}} \dd[3]{\bf r} \\
        &= \sum_{\bf k, \bf k'} \frac{2\pi}{Vk'^2}   \bra{\bf k} \underbrace{\frac{1}{V} \int  \dd[3]{\bf r}\ket{\bf r} \bra{\bf r}}_1 \ket{\bf k'} S_{\bf k}^*   \bra{\bf k'}  \ket{\chi }  S_{\bf k'}\\
        &= \sum_{\bf k, \bf k'} \frac{2\pi}{Vk'^2}   \delta_{\bf k' \bf k} S_{\bf k}^*   \bra{\bf k'}  \ket{\chi }  S_{\bf k'}\\
        &= \boxed{\sum_{ \bf k} \frac{2\pi}{Vk^2}  \bra{\bf k}  \ket{\chi } \abs{ S_{\bf k}}^2}
    \end{align*}
    which is the Ewald expression of reciprocal space electrostatic energy. Given a desired relative error bound, only finite numbers \(K_1, K_2, K_3\) of \(\bf k\) need to be taken into account.
    \subsection{Particle Mesh Ewald}
    The Ewald structure factor calculated in \cref{eq:esf} is computationally complex in that it scales as \(O(N_a^2)\), where \(N_a\) is the number of atoms in a unit cell of the system. The Particle Mesh Ewald avoids this complexity by ``meshing'' the multipoles onto grid points.

    \section{Formally deriving derivatives of the mesh function}
    The weight function \(\theta\) is differentiated up to \(l_\text{max}\) times. We need compact expressions for the evaluation of its derivatives. 
    \[
        \theta = \prod_{d = 1}^3 M_n \pqty{ N_d ( \mathbf r - \mathbf R_a) \dotproduct \bf a^*_d + \frac{n}{2} }
    \]
    Denoting \(u_d = N_d ( \mathbf r - \mathbf R_a) \dotproduct \bf a^*_d + \frac{n}{2} \) and \(\partial_i \equiv \pdv[]{}{u_i}\), we have for first derivative
    \begin{align*}
        \partial_i \theta &= \dv[]{\theta}{(\ln \theta)} \pdv[]{(\ln \theta)}{u_i}\\
        &= \theta \partial_i \pqty{\sum_d \ln M_n(u_d) }\\
        &= \theta \sum_d \partial_i \ln M_n(u_d) \\
        &= \theta \frac{M'_n(u_i)}{M_n(u_i)} 
    \end{align*}
    where \(M'_n, M''_n\) are derivatives of \(M_n\) derived in \texttt{derive\_theta.py}. The above expression can be easily verified to be correct by writing it in the following form \[
        \pdv{(M_n(u_1) M_n(u_2) M_n(u_3))}{u_i} = M_n(u_1) M_n(u_2) M_n(u_3) \frac{M'_n (u_i)}{M_n(u_i)}.
    \]
    In numpy code, the fraction is a elementwise division.
    
    Moving on to second derivative, \begin{align*}
        \partial_j \partial_i \theta &= \partial_j \pqty{\theta \partial_i \ln \theta}\\
        &= \frac{\pqty{\partial_i \theta} \pqty{\partial_j \theta}}{\theta} + \theta \partial_j \partial_i \ln M_n (u_i)\\
        &= \frac{\pqty{\partial_i \theta} \pqty{\partial_j \theta}}{\theta} + \theta \delta_{ij} \bqty{\frac{M''_n (u_i)}{M_n(u_i)} - \pqty{\frac{M'_n(u_i)}{M_n(u_i)} }^2}\\
        &= \theta \delta_{ij} \bqty{\frac{M''_n (u_i)}{M_n(u_i)} } + \frac{\pqty{\partial_i \theta} \pqty{\partial_j \theta} - \delta_{ij} \pqty{\partial_i \theta}^2 }{\theta} 
    \end{align*}
    where the first term is diagonal and the second term is off-diagonal, as we would have inspected if we wrote \[
        \pdv{\theta}{u_i}{u_j} = \begin{pmatrix} 
            M''_1 M_2 M_3 & M'_1 M'_2 M_3 & M'_1 M_2 M'_3\\
            M'_1 M'_2 M_3 & M_1 M''_2 M_3 & M_1 M'_2 M'_3\\
            M'_1 M_2 M'_3 & M_1 M'_2 M'_3 & M_1 M_2 M''_3
        \end{pmatrix}_{ij}
    \]

    Third derivative:
    \begin{align*}
        \partial_k \partial_j \partial_i \theta 
        &= \partial_k \partial_j (\theta \partial_i \ln \theta)\\
        &= \frac{(\partial_k \partial_j \theta) \partial_i \theta}{\theta} +  
    \end{align*}
    
    Derivations demonstrated above should allow us to express higher order derivatives of \(\theta\) in terms of direct products of \(M_d\) with more confidence and ease.

    To get derivatives with respect to \(\bf r\), chain rule can be employed \[
        \pdv[]{}{r_i} = \sum_j \pdv[]{u_j}{r_i} \pdv[]{}{u_j} = \sum_j - N_j A^*_{ji} \pdv{u_j}
    \]
    where \(A^*_{ji} \hat{\bf e}_i = \bf a^*_j\). Since \( - N_j A^*_{ji}\) are constant Jacobians, the transform is easy in arbitrary order of derivative.

\section{Implementation details}




\end{document}